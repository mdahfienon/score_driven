\documentclass{article}
\usepackage{graphicx} % Required for inserting images
\usepackage[a4paper, total={6in, 10in}]{geometry}
%usepackage{comment}

\title{\textsc{Dynamic risk parity portfolio with independent component analysis}}
%\subtitle{Dynamic risk parity portfolio with independent component analysis}
\author{\textsc{Mathias Dah Fienon} \\ \\  Supervised by : \textsc{Christian Hafner}}
\date{December 2024}

\begin{document}

\maketitle

\section*{Introduction}
%\begin{comment}
%    In the growth framework of financial markets and institutions, investors and risk manager need to
%\end{comment}  

In risk parity scenarios, traditional risk parity aims to equalize the risk contribution of each asset within a portfolio, rather than weighting based on expected returns or even fixed allocations. These traditional approaches do not consider the dynamic exhibited by modern financial markets, particularly during periods of stress. In that, modern technics suggest incorporating these dynamics by referring to dynamic risk-parity portfolio selection, where the portfolio weights are adjusted over time based on evolving market conditions. This approach involves rebalancing more frequently and may use various indicators (e.g. volatility, correlations, market trends) to adjust allocations and adapt to shifts in risk levels across assets. The dynamic adjustment aims to optimize the portfolio risk balance in response to observed changes in the market.
This research proposes a novel dynamic risk parity framework based on Independent Component Analysis (ICA), which decomposes portfolio returns into statistically independent sources of risk. By dynamically adapting portfolio weights based on independent risk factors, the model aims to achieve superior diversification, resilience to tail risks, and better performance in varying market conditions.

\section*{Goals of the research}
\section*{State of the art}
\section*{Research project}
The proposed research seeks to address the following objectives:
\begin{itemize}
    \item Develop a robust theoretical framework for integrating ICA into the construction of a portfolio of risk parity.
    \item Design a dynamic portfolio rebalancing mechanism that optimizes risk contributions based on changing market conditions.
    \item Compare the performance of the proposed model against traditional risk parity and factor-based portfolio optimization strategies in terms of diversification, volatility reduction, and drawdown control.
    \item Apply the model to specialized domains such as insurance and banking investment portfolios, highlighting its utility in mitigating systemic risks and quantifying cyber risks.
\end{itemize}
\section*{Work plan}
\section*{References}
\section*{Research objectives}

\end{document}
